\documentclass[]{article}
\usepackage{lmodern}
\usepackage{amssymb,amsmath}
\usepackage{ifxetex,ifluatex}
\usepackage{fixltx2e} % provides \textsubscript
\ifnum 0\ifxetex 1\fi\ifluatex 1\fi=0 % if pdftex
  \usepackage[T1]{fontenc}
  \usepackage[utf8]{inputenc}
\else % if luatex or xelatex
  \ifxetex
    \usepackage{mathspec}
  \else
    \usepackage{fontspec}
  \fi
  \defaultfontfeatures{Ligatures=TeX,Scale=MatchLowercase}
\fi
% use upquote if available, for straight quotes in verbatim environments
\IfFileExists{upquote.sty}{\usepackage{upquote}}{}
% use microtype if available
\IfFileExists{microtype.sty}{%
\usepackage{microtype}
\UseMicrotypeSet[protrusion]{basicmath} % disable protrusion for tt fonts
}{}
\usepackage[margin=1in]{geometry}
\usepackage{hyperref}
\hypersetup{unicode=true,
            pdftitle={Linear Algebra},
            pdfauthor={Marc-André Désautels},
            pdfborder={0 0 0},
            breaklinks=true}
\urlstyle{same}  % don't use monospace font for urls
\usepackage{color}
\usepackage{fancyvrb}
\newcommand{\VerbBar}{|}
\newcommand{\VERB}{\Verb[commandchars=\\\{\}]}
\DefineVerbatimEnvironment{Highlighting}{Verbatim}{commandchars=\\\{\}}
% Add ',fontsize=\small' for more characters per line
\usepackage{framed}
\definecolor{shadecolor}{RGB}{248,248,248}
\newenvironment{Shaded}{\begin{snugshade}}{\end{snugshade}}
\newcommand{\AlertTok}[1]{\textcolor[rgb]{0.94,0.16,0.16}{#1}}
\newcommand{\AnnotationTok}[1]{\textcolor[rgb]{0.56,0.35,0.01}{\textbf{\textit{#1}}}}
\newcommand{\AttributeTok}[1]{\textcolor[rgb]{0.77,0.63,0.00}{#1}}
\newcommand{\BaseNTok}[1]{\textcolor[rgb]{0.00,0.00,0.81}{#1}}
\newcommand{\BuiltInTok}[1]{#1}
\newcommand{\CharTok}[1]{\textcolor[rgb]{0.31,0.60,0.02}{#1}}
\newcommand{\CommentTok}[1]{\textcolor[rgb]{0.56,0.35,0.01}{\textit{#1}}}
\newcommand{\CommentVarTok}[1]{\textcolor[rgb]{0.56,0.35,0.01}{\textbf{\textit{#1}}}}
\newcommand{\ConstantTok}[1]{\textcolor[rgb]{0.00,0.00,0.00}{#1}}
\newcommand{\ControlFlowTok}[1]{\textcolor[rgb]{0.13,0.29,0.53}{\textbf{#1}}}
\newcommand{\DataTypeTok}[1]{\textcolor[rgb]{0.13,0.29,0.53}{#1}}
\newcommand{\DecValTok}[1]{\textcolor[rgb]{0.00,0.00,0.81}{#1}}
\newcommand{\DocumentationTok}[1]{\textcolor[rgb]{0.56,0.35,0.01}{\textbf{\textit{#1}}}}
\newcommand{\ErrorTok}[1]{\textcolor[rgb]{0.64,0.00,0.00}{\textbf{#1}}}
\newcommand{\ExtensionTok}[1]{#1}
\newcommand{\FloatTok}[1]{\textcolor[rgb]{0.00,0.00,0.81}{#1}}
\newcommand{\FunctionTok}[1]{\textcolor[rgb]{0.00,0.00,0.00}{#1}}
\newcommand{\ImportTok}[1]{#1}
\newcommand{\InformationTok}[1]{\textcolor[rgb]{0.56,0.35,0.01}{\textbf{\textit{#1}}}}
\newcommand{\KeywordTok}[1]{\textcolor[rgb]{0.13,0.29,0.53}{\textbf{#1}}}
\newcommand{\NormalTok}[1]{#1}
\newcommand{\OperatorTok}[1]{\textcolor[rgb]{0.81,0.36,0.00}{\textbf{#1}}}
\newcommand{\OtherTok}[1]{\textcolor[rgb]{0.56,0.35,0.01}{#1}}
\newcommand{\PreprocessorTok}[1]{\textcolor[rgb]{0.56,0.35,0.01}{\textit{#1}}}
\newcommand{\RegionMarkerTok}[1]{#1}
\newcommand{\SpecialCharTok}[1]{\textcolor[rgb]{0.00,0.00,0.00}{#1}}
\newcommand{\SpecialStringTok}[1]{\textcolor[rgb]{0.31,0.60,0.02}{#1}}
\newcommand{\StringTok}[1]{\textcolor[rgb]{0.31,0.60,0.02}{#1}}
\newcommand{\VariableTok}[1]{\textcolor[rgb]{0.00,0.00,0.00}{#1}}
\newcommand{\VerbatimStringTok}[1]{\textcolor[rgb]{0.31,0.60,0.02}{#1}}
\newcommand{\WarningTok}[1]{\textcolor[rgb]{0.56,0.35,0.01}{\textbf{\textit{#1}}}}
\usepackage{graphicx,grffile}
\makeatletter
\def\maxwidth{\ifdim\Gin@nat@width>\linewidth\linewidth\else\Gin@nat@width\fi}
\def\maxheight{\ifdim\Gin@nat@height>\textheight\textheight\else\Gin@nat@height\fi}
\makeatother
% Scale images if necessary, so that they will not overflow the page
% margins by default, and it is still possible to overwrite the defaults
% using explicit options in \includegraphics[width, height, ...]{}
\setkeys{Gin}{width=\maxwidth,height=\maxheight,keepaspectratio}
\IfFileExists{parskip.sty}{%
\usepackage{parskip}
}{% else
\setlength{\parindent}{0pt}
\setlength{\parskip}{6pt plus 2pt minus 1pt}
}
\setlength{\emergencystretch}{3em}  % prevent overfull lines
\providecommand{\tightlist}{%
  \setlength{\itemsep}{0pt}\setlength{\parskip}{0pt}}
\setcounter{secnumdepth}{0}
% Redefines (sub)paragraphs to behave more like sections
\ifx\paragraph\undefined\else
\let\oldparagraph\paragraph
\renewcommand{\paragraph}[1]{\oldparagraph{#1}\mbox{}}
\fi
\ifx\subparagraph\undefined\else
\let\oldsubparagraph\subparagraph
\renewcommand{\subparagraph}[1]{\oldsubparagraph{#1}\mbox{}}
\fi

%%% Use protect on footnotes to avoid problems with footnotes in titles
\let\rmarkdownfootnote\footnote%
\def\footnote{\protect\rmarkdownfootnote}

%%% Change title format to be more compact
\usepackage{titling}

% Create subtitle command for use in maketitle
\newcommand{\subtitle}[1]{
  \posttitle{
    \begin{center}\large#1\end{center}
    }
}

\setlength{\droptitle}{-2em}
  \title{Linear Algebra}
  \pretitle{\vspace{\droptitle}\centering\huge}
  \posttitle{\par}
  \author{Marc-André Désautels}
  \preauthor{\centering\large\emph}
  \postauthor{\par}
  \predate{\centering\large\emph}
  \postdate{\par}
  \date{15 février 2018}

\usepackage{xfrac}

\begin{document}
\maketitle

\hypertarget{utilisation-de-r-pour-produire-du-code-latex-en-algebre-lineaire}{%
\section{\texorpdfstring{Utilisation de \texttt{R} pour produire du code
LaTeX en algèbre
linéaire}{Utilisation de R pour produire du code LaTeX en algèbre linéaire}}\label{utilisation-de-r-pour-produire-du-code-latex-en-algebre-lineaire}}

\hypertarget{initialisation-de-la-librairie}{%
\subsection{Initialisation de la
librairie}\label{initialisation-de-la-librairie}}

Nous devons installer la librairie. Si vous n'avez pas la librairie
\texttt{devtools}, vous devez l'installer.

\begin{Shaded}
\begin{Highlighting}[]
\KeywordTok{install.packages}\NormalTok{(}\StringTok{"devtools"}\NormalTok{)}
\end{Highlighting}
\end{Shaded}

Vous installer ensuite la librairie à l'aide de la commande suivante:

\begin{Shaded}
\begin{Highlighting}[]
\NormalTok{devtools}\OperatorTok{::}\KeywordTok{install_github}\NormalTok{(}\StringTok{"desautm/linalgr"}\NormalTok{)}
\end{Highlighting}
\end{Shaded}

Vous pouvez charger la librairie:

\begin{Shaded}
\begin{Highlighting}[]
\KeywordTok{library}\NormalTok{(linalgr)}
\end{Highlighting}
\end{Shaded}

\hypertarget{affichage-de-matrices}{%
\subsection{Affichage de matrices}\label{affichage-de-matrices}}

Nous allons définir quelques matrices:

\begin{Shaded}
\begin{Highlighting}[]
\NormalTok{m <-}\StringTok{ }\DecValTok{5}
\NormalTok{n <-}\StringTok{ }\DecValTok{5}
\NormalTok{A <-}\StringTok{ }\KeywordTok{matrix}\NormalTok{(}\KeywordTok{sample}\NormalTok{(}\OperatorTok{-}\DecValTok{10}\OperatorTok{:}\DecValTok{10}\NormalTok{, m}\OperatorTok{*}\NormalTok{n, }\DataTypeTok{replace =} \OtherTok{TRUE}\NormalTok{), m, n)}
\NormalTok{B <-}\StringTok{ }\KeywordTok{matrix}\NormalTok{(}\KeywordTok{sample}\NormalTok{(}\OperatorTok{-}\DecValTok{10}\OperatorTok{:}\DecValTok{10}\NormalTok{, m, }\DataTypeTok{replace =} \OtherTok{TRUE}\NormalTok{), m, }\DecValTok{1}\NormalTok{)}
\end{Highlighting}
\end{Shaded}

Voici l'affichage directement avec \texttt{R}:

\begin{Shaded}
\begin{Highlighting}[]
\NormalTok{A}
\end{Highlighting}
\end{Shaded}

\begin{verbatim}
##      [,1] [,2] [,3] [,4] [,5]
## [1,]  -10    6   -4   -9   -9
## [2,]   -3   -5   -3   -5   -4
## [3,]    0   -5   -6   -1  -10
## [4,]   -7    6   -5    9   -1
## [5,]    0    5    9  -10   -1
\end{verbatim}

\begin{Shaded}
\begin{Highlighting}[]
\NormalTok{B}
\end{Highlighting}
\end{Shaded}

\begin{verbatim}
##      [,1]
## [1,]    7
## [2,]    5
## [3,]    0
## [4,]    9
## [5,]    1
\end{verbatim}

Voici l'affichage en utilisant la librairie:

\begin{Shaded}
\begin{Highlighting}[]
\KeywordTok{mat2latex}\NormalTok{(A)}
\KeywordTok{mat2latex}\NormalTok{(B)}
\end{Highlighting}
\end{Shaded}

\[
 \left[
\begin{array}{rrrrr}
-10 & 6 & -4 & -9 & -9 \\ 
-3 & -5 & -3 & -5 & -4 \\ 
0 & -5 & -6 & -1 & -10 \\ 
-7 & 6 & -5 & 9 & -1 \\ 
0 & 5 & 9 & -10 & -1 \\ 
\end{array}
\right] 
\]\[
 \left[
\begin{array}{r}
7 \\ 
5 \\ 
0 \\ 
9 \\ 
1 \\ 
\end{array}
\right] 
\]

\hypertarget{affichage-de-sel}{%
\section{Affichage de SEL}\label{affichage-de-sel}}

Nous pouvons afficher des SEL

\begin{Shaded}
\begin{Highlighting}[]
\KeywordTok{sel2latex}\NormalTok{(A,B, }\DataTypeTok{variables =} \StringTok{"xi"}\NormalTok{)}
\end{Highlighting}
\end{Shaded}

\[
 \begin{array}{rrrrrrrrrrr}
-10 x_{1} & + & 6 x_{2} & - & 4 x_{3} & - & 9 x_{4} & - & 9 x_{5} & = & 7 \\ 
-3 x_{1} & - & 5 x_{2} & - & 3 x_{3} & - & 5 x_{4} & - & 4 x_{5} & = & 5 \\ 
& - & 5 x_{2} & - & 6 x_{3} & - & x_{4} & - & 10 x_{5} & = & 0 \\ 
-7 x_{1} & + & 6 x_{2} & - & 5 x_{3} & + & 9 x_{4} & - & x_{5} & = & 9 \\ 
& & 5 x_{2} & + & 9 x_{3} & - & 10 x_{4} & - & x_{5} & = & 1 \\ 
\end{array} 
\]

\begin{Shaded}
\begin{Highlighting}[]
\KeywordTok{sel2latex}\NormalTok{(A,B, }\DataTypeTok{sel =} \OtherTok{FALSE}\NormalTok{, }\DataTypeTok{variables =} \StringTok{"a"}\NormalTok{)}
\end{Highlighting}
\end{Shaded}

\[
 \left[
\begin{array}{rrrrr}
-10 & 6 & -4 & -9 & -9 \\ 
-3 & -5 & -3 & -5 & -4 \\ 
0 & -5 & -6 & -1 & -10 \\ 
-7 & 6 & -5 & 9 & -1 \\ 
0 & 5 & 9 & -10 & -1 \\ 
\end{array}
\right] \left[
\begin{array}{c} 
a \\ 
b \\ 
c \\ 
d \\ 
e \\
\end{array}
\right] = \left[
\begin{array}{r}
7 \\ 
5 \\ 
0 \\ 
9 \\ 
1 \\ 
\end{array}
\right] 
\]

Avec des fractions décimales

\begin{Shaded}
\begin{Highlighting}[]
\KeywordTok{sel2latex}\NormalTok{(A}\OperatorTok{/}\DecValTok{3}\NormalTok{,B, }\DataTypeTok{variables =} \StringTok{"xi"}\NormalTok{, }\DataTypeTok{digits =} \DecValTok{3}\NormalTok{)}
\end{Highlighting}
\end{Shaded}

\[
 \begin{array}{rrrrrrrrrrr}
-3.333 x_{1} & + & 2.000 x_{2} & - & 1.333 x_{3} & - & 3.000 x_{4} & - & 3.000 x_{5} & = & 7 \\ 
-1.000 x_{1} & - & 1.667 x_{2} & - & 1.000 x_{3} & - & 1.667 x_{4} & - & 1.333 x_{5} & = & 5 \\ 
& - & 1.667 x_{2} & - & 2.000 x_{3} & - & 0.333 x_{4} & - & 3.333 x_{5} & = & 0 \\ 
-2.333 x_{1} & + & 2.000 x_{2} & - & 1.667 x_{3} & + & 3.000 x_{4} & - & 0.333 x_{5} & = & 9 \\ 
& & 1.667 x_{2} & + & 3.000 x_{3} & - & 3.333 x_{4} & - & 0.333 x_{5} & = & 1 \\ 
\end{array} 
\]

Avec des fractions ordinaires

\begin{Shaded}
\begin{Highlighting}[]
\KeywordTok{sel2latex}\NormalTok{(A}\OperatorTok{/}\DecValTok{3}\NormalTok{,B, }\DataTypeTok{variables =} \StringTok{"xi"}\NormalTok{, }\DataTypeTok{style =} \StringTok{"inline"}\NormalTok{)}
\end{Highlighting}
\end{Shaded}

\[
 \begin{array}{rrrrrrrrrrr}
-10/3 x_{1} & + & 2 x_{2} & - & 4/3 x_{3} & - & 3 x_{4} & - & 3 x_{5} & = & 7 \\ 
-x_{1} & - & 5/3 x_{2} & - & x_{3} & - & 5/3 x_{4} & - & 4/3 x_{5} & = & 5 \\ 
& - & 5/3 x_{2} & - & 2 x_{3} & - & 1/3 x_{4} & - & 10/3 x_{5} & = & 0 \\ 
-7/3 x_{1} & + & 2 x_{2} & - & 5/3 x_{3} & + & 3 x_{4} & - & 1/3 x_{5} & = & 9 \\ 
& & 5/3 x_{2} & + & 3 x_{3} & - & 10/3 x_{4} & - & 1/3 x_{5} & = & 1 \\ 
\end{array} 
\]

\begin{Shaded}
\begin{Highlighting}[]
\KeywordTok{sel2latex}\NormalTok{(A}\OperatorTok{/}\DecValTok{3}\NormalTok{,B, }\DataTypeTok{variables =} \StringTok{"xi"}\NormalTok{, }\DataTypeTok{style =} \StringTok{"sfrac"}\NormalTok{)}
\end{Highlighting}
\end{Shaded}

\[
 \begin{array}{rrrrrrrrrrr}
-\sfrac{10}{3} x_{1} & + & 2 x_{2} & - & \sfrac{4}{3} x_{3} & - & 3 x_{4} & - & 3 x_{5} & = & 7 \\ 
-x_{1} & - & \sfrac{5}{3} x_{2} & - & x_{3} & - & \sfrac{5}{3} x_{4} & - & \sfrac{4}{3} x_{5} & = & 5 \\ 
& - & \sfrac{5}{3} x_{2} & - & 2 x_{3} & - & \sfrac{1}{3} x_{4} & - & \sfrac{10}{3} x_{5} & = & 0 \\ 
-\sfrac{7}{3} x_{1} & + & 2 x_{2} & - & \sfrac{5}{3} x_{3} & + & 3 x_{4} & - & \sfrac{1}{3} x_{5} & = & 9 \\ 
& & \sfrac{5}{3} x_{2} & + & 3 x_{3} & - & \sfrac{10}{3} x_{4} & - & \sfrac{1}{3} x_{5} & = & 1 \\ 
\end{array} 
\]


\end{document}
